\documentclass[12pt, a4paper]{IEEEtran}
    \usepackage{amsmath}
    \usepackage[pdftex]{graphicx}
    \usepackage{float}
    \usepackage{ tipa }
    \usepackage{ dsfont }
    %设置四周
    \usepackage[left=1in, right=1in]{geometry}
    %设置字体
    \usepackage{times}
    \usepackage{mathptmx}
    %设置行距
    \linespread{0.7}
    %开始
    
    \begin{document}
    \centerline{\textbf{Name: Chen Chen}}
    
    \centerline{\textbf{Dis 2A}}
    \begin{flushleft}
    
    
    \section*{4.6}
    \subsection*{2}
    $y_1(t)=cos2t$ and $y_2(t)=sin2t$\\
    $y=v_1cos2t+v_2sin2t$\\
    $y'=v_1'cos2t+v_2'sin2t-2v_1cos2t+2v_2sin2t$\\
    Because $v_1'cos2t+v_2sin2t=0$,\\
    $y'=-2v_1cos2t+2v_2sin2t$\\
    $y''=-2v_1'cos2t+2v_2'sin2t-4(v_1cos2t+v_2sin2t)$\\
    $y''+4y=-2v_1'cos2t+2v_2'sin2t=sec2t$\\
    Because $v_1'cos2t+v_2sin2t=0$ and $-2v_1'cos2t+2v_2'sin2t=sec2t$\\
    $v_1'=-\frac{1}{2}tan2t$ and $v_2'=\frac{1}{2}$\\
    $y=\frac{1}{4}cos2t\cdot In(cos2t)+\frac{t}{2}sin2t$\\
    
    \subsection*{7}
    Because $W(cost, sint)=1$,\\
    $x_p=v_1x_1+v_2x_2$\\
    $v_1'=-secttant+sint$\\
    $v_1=-sect-cost$\\
    $v_2'=sect-cost$\\
    $v_2=In|sect+tant|-sint$\\
    $x_p=-2+sint\cdot In|sect+tant|$\\
    
    \subsection*{15}
    $y''+\frac{3}{t}y'+\frac{1}{t^2}y=\frac{1}{t^3}$\\
    plug in $y_1(t)=t^{-1}$\\
    $2t^{-3}-\frac{3}{t}t^{-2}+\frac{1}{t^2}t^{-1}=0$\\
    $y_2(t)'=t^{-2}(1-Int)$ and $y_2(t)''=t^{-3}(-3+2Int)$\\
    plug in, $t^{-3}(-3+2Int)+3t^{-3}(1-Int)+t^{-3}Int=0$\\
    $W(t^{-1}, t^{-1}Int)=t^{-3}$\\
    $v_1'=t^{-1}Int$, So $v_1=-\frac{1}{2}(Int)^2$\\
    $v_2'=t^{-1}$, So $v_2=Int$\\
    $y_p=\frac{1}{2}t^{-1}(Int)^2$\\
    $y(t)=C_1t^{-1}+C_2t^{-1}Int+\frac{1}{2}t^{-1}(Int)^2$\\
    $t>0$\\
    

    \section*{9}
    \subsection*{4}
    $A=\begin{pmatrix}
        -4&1\\
        -2&1
        \end{pmatrix}$\\
    $p(\lambda)=det(A-\lambda I)=\lambda^2+3\lambda-2$\\
    $\lambda_1=\frac{(-3-\sqrt{17})}{2}$, $\lambda_2=\frac{-3-\sqrt{17}}{2}$\\

    \subsection*{8}
    $A=\begin{pmatrix}
        -3&0\\
        0&-3\\
    \end{pmatrix}$\\
    $p(\lambda)=det(A-\lambda I)=(\lambda +3)^2$\\
    $\lambda=-3$

    \subsection*{10}
    $p(\lambda)=-(1-\lambda)\begin{vmatrix}
        3-\lambda&2\\
        -4&-3\lambda
    \end{vmatrix}=(\lambda-1)^2(\lambda+1)$\\
    $\lambda_1=-1$ and $\lambda_2=1$
    
    \subsection*{18}
    $p(\lambda)=det\begin{pmatrix}
        -3-\lambda&-4\\
        2&3-\lambda
    \end{pmatrix}=(\lambda+1)(\lambda-1)$\\
    $\lambda_1=-1$ and $\lambda_2=1$\\
    $A+I=\begin{pmatrix}
        -2&-4\\
        2&4
    \end{pmatrix}$\\
    $y_1(t)=e^{-t}\dbinom{-2}{1}$\\
    $A-I=\begin{pmatrix}
        -4&-4\\
        2&2
    \end{pmatrix}$\\
    $y_2(t)=e^t\dbinom{1}{-1}$\\
    
    \subsection*{24}
    $p(\lambda)=(3+\lambda)\begin{vmatrix}
        -5-\lambda&-6\\
        4&5-\lambda
    \end{vmatrix}=(\lambda+3)(\lambda+1)(\lambda-1)$\\
    $\lambda_1=-3$ and $\lambda_2=-1$ and $\lambda_3=1$\\
    $A+3I=\begin{pmatrix}
        -2&0&-6\\
        26&0&38\\
        4&0&8
    \end{pmatrix}$\\
    $y_1(t)=e^{-3t}\begin{pmatrix}
        0\\
        1\\
        0
    \end{pmatrix}$\\
    $A+I=\begin{pmatrix}
        -4&0&-6\\
        26&-2&38\\
        4&0&6
    \end{pmatrix}$\\
    $y_2(t)=e^{-t}\begin{pmatrix}
        -3\\
        -1\\
        2
    \end{pmatrix}$\\
    $A-I=\begin{pmatrix}
        -6&0&-6\\
        26&-4&38\\
        4&0&4\\
    \end{pmatrix}$\\
    $y_3(t)=e^t\begin{pmatrix}
        -3\\
        3\\
        1
    \end{pmatrix}$\\
    $det[y_1(0), y_2(0), y_3(0)]=1$




    \end{flushleft}
    \end{document}